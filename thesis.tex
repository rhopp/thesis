\documentclass[11pt,draft,oneside]{fithesis}
\usepackage[plainpages=false, pdfpagelabels]{hyperref}
\thesistitle{Tvorba dokumentu v XML}
\thesissubtitle{Bakalářská práce}
\thesisstudent{Jméno Příjmení}
\thesiswoman{false}
\thesisfaculty{fi}
\thesisyear{jaro 2003}
\thesisadvisor{Jméno Příjmení}

\begin{document}
\FrontMatter
\ThesisTitlePage

\begin{ThesisDeclaration}
\DeclarationText
\AdvisorName
\end{ThesisDeclaration}

\begin{ThesisThanks}
Zde bude uvedeno \uv{poděkování} ... 
\end{ThesisThanks}

Obdobně jako poděkování se mohou vysadit shrnutí a klíčová 
slova pomocí prostředí Thesisacti a ThesisKeyWordsi.

\MainMatter
\tableofcontents
\chapter*{Úvod}
Text ...

% Následují další kapitoly a podkapitoly, popřípadě závěr, dodatky, 
% seznam literatury či použitých obrázků nebo tabulek.

\bibliographystyle{plain}  % bibliografický styl 
\bibliography{mujbisoubor} % soubor s citovanými
                           % položkami bibliografie 

\end{document}
